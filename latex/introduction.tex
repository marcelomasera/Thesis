\section{Introduction}
With the globalization of financial markets, investors have access to a larger
investment universe giving them the opportunity to better diversify their
portfolio. A subclass of financial transactions (short selling and entering
into derivative contracts for example) requires the posting of margins, that is,
the deposit of collateral to attenuate the credit risk for the counterparty. If
such a transaction is performed in a foreign market, the required collateral
will most likely be in a currency different from the portfolio's benchmark
currency, creating foreign exchange risk for the global investor. This investor
might therefore want to maintain the posted collateral to the strict minimum.
This policy will however maximize the probability that the counterparty, either
the investor's broker or exchange, will require the posting of additional
collateral following an adverse price change, so-called a margin call.

Margin calls sometimes result in undesired operational costs for investors, they
are therefore to be avoided if possible. Not only counterparties may impose
penalties on margin calls, but the regulatory framework agreed upon by members
of the Basel Committee on Banking Supervision in the Basel III accords impose
tougher capital buffers and liquidity coverage on derivatives transactions
\citep{basel1,basel2}.

The purpose of this work is to use recent developments in the field of
econometrics to come up with a  rigorous solution to the problem of
multicurrency collateral management. In addition, the methodology proposed here
can be extended to many financial and risk management applications where an
optimization problem needs to be constrained by practical considerations
(transaction costs for example). 

The quest to a better understanding and forecasting of financial time series
resulted in a constant increase in the level of sophistication of econometrics
models. For univariate series, the canonical Brownian process
\citep{brown1828, bachelier1900, merton69} has often been substituted with more
recent models capturing empirical properties of the financial time series such
as heteroskedasticity and leptokurticity (fat tails). The most popular of these
models are AutoRegressive Conditionnal  Heteroskedasticity (ARCH) \citep{engle82}
and its extension the generalized ARCH (GARCH) \citep{bollerslev86}. In these
models, the variance, and hence the volatility, of the time series for a given
period is a direct function of the variance in the previous period, thus
accounting for volatility clustering. Furthermore, the processes generate data
with fatter tails than the normal density. Thanks to its parsimonious notation
(see equations \eqref{drift} and \eqref{volatility} in Section~\ref{modelling}),
GARCH has become the most popular ARCH model in practice, including this work.

Proper financial modeling requires the consideration of dependence
between time series. For this reason the development of multivariate models
quickly followed the one of univariate models. It is now widely accepted
\citep{cont01, yue12} that robust risk management requires the inclusion of
higher-order moments and co-moments in the time series of financial assets.
Not only do univariate distributions capturing higher moments need to be picked
to model single financial time series, but a model of dependence between the
different time series allowing for higher co-moments is primordial. Indeed,
traditional linear correlation between the returns of financial assets fails
to capture the ``correlation breakdown'' or ``assets boom alone but bust
together'' asymmetry observed in the markets. Models have been proposed to
extend the ARCH-type framework described in the  previous paragraphs to
multivariate settings. They all face the challenge of balancing sophistication
with parsimony, of being flexible enough to capture co-moment dynamics while
avoiding the curse of dimensionality. For a comprehensive survey of
multivariate GARCH models, refer to \cite{silvennoinen08} or \cite{laurent06}.

A promising family of multivariate models combines the tractability of univariate
processes with the power of copulas to capture the dependence between the marginal
process in a robust manner. A copula, in combination with the univariate marginal
distributions,  is sufficient to fully specify a multivariate distribution
function, as proven by \cite{sklar59}. The copula $C$ underlying the random
variables $X_1$, $X_2$, $\ldots$, $X_D$ is the joint cumulative distribution
function of the transformed variables $F_1(x_1), F_2(x_2), \ldots,  F_D(x_D)$,
where $F_i(x)=\mathbb{P}[X_i \leq x]$ are the marginal cumulative distribution
functions, and
\begin{displaymath}
C(u_1, \ u_2, \ \ldots \ , u_D)=\mathbb{P}[F_1(X_1) \leq u_1, \
F_2(X_2) \leq u_2, \ \ldots \ , F_D(X_D) \leq u_D].
\end{displaymath}
This work owes much to the findings of Chen Xiaohong and Fan Yanqin
\citep{chenfan06}, Andrew Patton \citep{patton00,patton06} and Bruno
R\'emillard \citep{Remillard11}. The basis of these articles is that
the dependence between the error terms of multivariate time series
is described not in term of traditional correlation but by a given
copula.

Unfortunately, copulas have often been used prior to the existence of
proper goodness-of-fit tests, that is, without properly testing whether
the data being modeled belongs to the chosen copula or not in a
statistically significant manner. One of the first instance of copulas
goodness-of-fit test for multivariate time series  can be found in
\cite{chenfan06}, however this test can only rank the appropriateness
of different copulas relative to each other, not in a statistically
absolute sense. Fortunately, \cite{genestetal09} and \cite{Remillard11}
applied the parametric bootstrapping technique to copulas allowing one
to do so in an intuitive, powerful manner. Its use with copulas evolved
naturally from previous applications to univariate and multivariate
distributions. The application of this test prevents the use of a copula
where inappropriate.

The tools developed in this work can also be applied to related problems:
balancing tracking error and transaction costs (see \cite{chanramkumar11}),
foreign currency risk and hedging costs (see \cite{campbell2010}), sales
and marketing costs, etc. In the case of balancing foreign exchange risk
on posted collateral and the costs associated to margin calls, global
investors have historically used heuristics or more sophisticated policies
to mitigate the two inconveniences. For literature on the subject, one
can refer to \cite{MillerOrr, higson10} for solutions to the similar
problem of optimal inventory management, to \cite{cotter01,lam04,kao10,longin99}
for solutions to the problem of setting the margins requirements
by a central counterparties (e.g. clearing houses) and to
\cite{fujii2010} for an example where the market participant has
the choice of collateral currency.

The remainder of this article is organized as follows. Section~\ref{modelling}
exhibits the framework used to model the dynamics of the asset prices and
exchange rates. Section~\ref{optimisation} overviews the optimization problem at
hand. Section~\ref{backtesting} presents the results of a backtest of the
proposed solution performed on a real data set. Section~\ref{conclusion}
concludes and proposes avenues of future research.
