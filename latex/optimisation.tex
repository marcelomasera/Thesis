\section{Constrained optimization}
\label{optimisation}
One difficulty in the problem at hand is to find a way to balance foreign exchange risk on
one hand and the costs associated to margin calls on the other. This exercise proves difficult
given that these two forces have different units. We posited that the costs associated to
margin calls are an increasing function of their frequency, and chose to minimize the foreign exchange
risk on the posted collateral conditionally to an arbitrary upper bound on the probability
of a margin call. \cite{chanramkumar11} offers an elegant framework
to a problem similar in nature to ours. Their goal is to balance trading costs associated
with the rebalancing of a given portfolio and tracking error. Their solution is to
minimize the trading costs subject to an imposed upper ceiling on the forecasted
tracking error. Our objective function described below is inspired from this approach.

We formulate the issue at hand as a one-period constrained optimization problem.
The first step consists in choosing a risk measure to assess the foreign exchange
risk. This work uses the expected tracking error (ETE), the Value-at-Risk (VaR) and
the Tail Conditional Expectation (TCE). The method consists in finding the cash
balances that minimize the (foreign exchange) risk measure subject to an arbitrary
tolerance for the probability of a margin call. A mathematical formulation of the
optimization objective is
\begin{equation}
\label{riskmeasure} \min_{\lambda_{1,t}, \ldots, \lambda_{D,t}}
R^{\alpha}\left( \sum_{i=1}^D \left( \lambda_{i,t} -
\lambda_{i,t}^{\ast} \right) \cdot Y_{i,t+1} \right),
\end{equation}
where
\begin{equation}
\label{riskmeasure2}
R^{\alpha}(X) = \left\{ \begin{array}{cl}
              \mathbb{E}\left[ \left| X\right| \right] & \mbox{ for expected tracking error}, \\
              -\mathbb{Q}_{\alpha}(X) & \mbox{ for Value-at-Risk}, \\
              -\mathbb{E}[X|X\leq \mathbb{Q}_{\alpha}(X)] & \mbox{ for Tail Conditional Expectation}. \\
       \end{array} \right.
\end{equation}
where $\mathbb{Q}_{\alpha}(X)$, $\alpha \ \in \ [0,1]$ is a  quantile of $X$
such that $P(X\leq\mathbb{Q}_{\alpha}(X))=\alpha$. In words,
equation~(\ref{riskmeasure}) finds the collateral in each
currency $\lambda_{1,t}, \ldots, \lambda_{D,t}$  above the minimum
margins requirements imposed by the counterparty
$\lambda_{1,t}^{\ast}$, $\ldots$, $\lambda_{D,t}^{\ast}$ that
minimizes one of the risk measures $R^{\alpha}(\cdot)$ described in
equation~(\ref{riskmeasure2}) against the changes in the exchange
rates $Y_{1,t+1}$, $\ldots$, $Y_{D,t+1}$.
The first constraint of the optimization is
\begin{displaymath}
\lambda_{i,t}^{\ast}\leq\lambda_{i,t}\leq\infty,\ i=1,\ldots,D,
\end{displaymath}
that is, each posted collateral $\lambda_{i,t}$ must be between the
minimum margin requirement $\lambda_{i,t}^{\ast}$ and infinity.  The
second constraint of the optimization is
\begin{displaymath}
%\label{probmargincall}
\mathbb{P} \left( \left( \prod_{i=1}^D \mathds{1}\left(
\lambda_{i,t} + \mbox{PnL}_{i,t+1}\geq
\lambda_{i,t}^{\ast}\right)\right)=0 \right) \leq P_{\mbox{tol}},
\end{displaymath}
where
\begin{displaymath}
%\label{pnl}
\mbox{PnL}_{i,t+1}=\sum_{j=1}^{n_{i,t}} {}_i\omega_{j,t} \cdot {}_i
W_{j,t+1}.
\end{displaymath}
Here, $P_{\mbox{tol}}$ is an arbitrary tolerance for the probability
of a margin call,  $n_{i,t}$ is the number of different assets of
the $i^{th}$ currency held,  ${}_i\omega_{j,t}$ is the number of the
$j^{th}$ asset of the  $i^{th}$ currency held and ${}_i W_{j,t+1}$
is the change in the $j^{th}$ asset of the $i^{th}$ currency between
time $t$ and $t+1$. To make things clear, the only random variables
in the model are the $Y_{i,t}$ and the ${}_i W_{j,t+1}$.
Furthermore, these random variables are not independent of each
other. That is, the spot exchange rates are not independent of the
futures contracts prices and vice versa.

The proposed methodology for the optimization is as follow: First,
univariate process are fitted to each time-series of log returns of
exchange rates and future prices. This step often involves the
calibration of parameters with the maximum likelihood estimation
method. Second, goodness-of-fit tests of the selected model are
performed on each time-series. If the test result is negative for a
given time-series of log returns, then another univariate model is
chosen for that specific time-series and the process is started anew
from the first step. Again, different time-series can have different
univariate models. Third, a parametric multivariate copula is fitted
to the normalized ranks of the standardized residuals obtained in
the first step. Fourth, a goodness-of-fit test as put forward in
Appendix~\ref{GOF} is performed on the copula-based dynamic model
with respect to the sample of time-series. If the test result is
negative, another copula is chosen and the process is restarted from
the third step. At this point, we know that the model is appropriate
for the data (at least in the statistical sense of the term). Here
is where Monte Carlo methods come into play. Fifth, a large number
$N$ of $D$-dimensional $U[0,1]$ realizations is drawn from the
parametric copula, were $D$ is the number of time-series modeled.
Most modern statistical software packages provide tools to perform
this step for elliptical copulas. One can refer to \cite{marshallolkin88}
and to \cite{melchiori06} for Archimedean copula random variable
gerneration. Sixth, the simulated standardized residuals
of each univariate process are obtained by entering the uniform
drawings from the previous step into their respective inverse
cumulative distribution function. Seventh, the simulated
realizations of the log returns are obtained by combining of the
random standardized residuals and the computed deterministic part of
each univariate process. The final step consists in using
constrained numerical optimization to find the values of the control
variables that optimize the objective function while satisfying the
constraint function. For us, the control variables are the amount of
collateral held in each currency, the objective function is the risk
measure (tracking error, Var and TCE) and the constraint function is
the probability of a margin call. Notice that both the objective and
constraint are functions of the control variables (the amount of
collateral held in each foreign currency) and the
realizations from the Monte Carlo simulation (the log returns of
exchange rates and future prices). The objective and constraint
functions being nonlinear, we obtained the best results using the
interior point optimization method, closely followed by the active
set algorithm.
