\subsection{Alternative strategies}
\label{altstrats}
In order to access the validity of our model, we backtested it on
the dataset presented in the previous section. We also backtested
two other strategies on the same dataset to contrast their
performances.

The first, ``naive'' strategy simply consists in keeping twice the compulsory
collateral in the cash accounts at all times. That is, the collateral posted
in the accounts of the different currencies is brought back to twice the mandatory
minimum every day according to that day's movements in futures contracts prices
and holdings.

The second strategy consists in modeling the returns of the futures
contracts and exchange rates with a static multivariate normal
distribution as in the Markowitz framework. Doing so allows us to
gauge the accuracy added by factoring in stochastic volatility and
higher co-moments as done by the model proposed in
Section~\ref{modelling}.


\subsection{Calibration}
A buffer of 500 business days (approximately 2 years) is used at
the beginning of the sample to calibrate the marginal processes
and the dependence copula
parameters. The
means and covariance matrix for the static multivariate strategy
is also computed with this buffer. Both models are recalibrated
every day using all data from the beginning of the sample (i.e.
with an extending window).

The tolerance for the probability of a margin call $P_{tol}$ was
arbitrarily set to 0.05, which was also the level chosen for the
quantile $\alpha$ of the VaR and TCE measures.

Once at the beginning and then every 250 business days, with an
extending window, the goodness-of-fit tests described in
Appendix~\ref{GOF} were performed on both the marginal processes and
the dependence copulas. The results of the goodness-of-fit tests for
the AR(1)-GARCH(1,1) with Gaussian residuals, AR(2)-GARCH(2,2) with
Gaussian residuals, AR(1)-GARCH(1,1) with Student residuals and
different dependence copulas are displayed in
tables~\ref{ar1garch11}, \ref{ar2garch22}, \ref{ar1garch11_t} and
\ref{copulas} respectively. Clearly, a combination of
AR(1)-GARCH(1,1) with Student residuals and the Student copula is
the only appropriate model from a statistical point of view for the
time-series at hand.
\begin{table}[p]
    \begin{adjustwidth}{-1in}{-1in}
    \centering
    \input{ar1garch11.tex}
    \end{adjustwidth}
    \caption[GOF tests of AR(1)-GARCH(1,1) with Gaussian residuals]{\emph{p}-values from the goodness-of-fit tests of AR(1)-GARCH(1,1) with Gaussian residuals on the marginal processes. The number of bootstrapped samples is $N=100$.}
    \label{ar1garch11}
\end{table}
\begin{table}[p]
    \begin{adjustwidth}{-1in}{-1in}
    \centering
    \begin{scriptsize}
\begin{tabular}{lccccccc}
&\textbf{Feb06}&\textbf{Mar07}&\textbf{Apr08}&\textbf{Apr09}&\textbf{May10}&\textbf{Jun11}&\textbf{Jun12}\\\hline
\textbf{Japanese 10 Yr Future Mini}&0.06&0.10&0.02&0.00&0.00&0.00&0.00\\
\textbf{Can 10 Yr Future}&0.12&0.48&0.39&0.01&0.00&0.00&0.00\\
\textbf{Euro Bund Future}&0.00&0.00&0.00&0.00&0.00&0.00&0.00\\
\textbf{AUD 10 Yr Future}&0.01&0.00&0.00&0.00&0.00&0.00&0.00\\
\textbf{AU 1-3 year Future}&0.04&0.01&0.00&0.00&0.00&0.00&0.00\\
\textbf{AUD/USD}&0.33&0.09&0.00&0.00&0.00&0.00&0.00\\
\textbf{CAD/USD}&0.09&0.35&0.11&0.00&0.00&0.00&0.00\\
\textbf{EUR/USD}&0.01&0.00&0.00&0.00&0.00&0.00&0.00\\
\textbf{JPY/USD}&0.07&0.01&0.00&0.00&0.00&0.00&0.00\\
\end{tabular}
\end{scriptsize}

    \end{adjustwidth}
    \caption[GOF tests of AR(2)-GARCH(2,2) with Gaussian residuals]{\emph{p}-values from the goodness-of-fit tests of AR(2)-GARCH(2,2) with Gaussian residuals on the marginal processes. The number of bootstrapped samples is $N=100$}
    \label{ar2garch22}
\end{table}
\begin{table}[p]
    \begin{adjustwidth}{-1in}{-1in}
    \centering
    \begin{scriptsize}
\begin{tabular}{lccccccc}
&\textbf{Feb06}&\textbf{Mar07}&\textbf{Apr08}&\textbf{Apr09}&\textbf{May10}&\textbf{Jun11}&\textbf{Jun12}\\\hline
\textbf{Japanese 10 Yr Future Mini}&0.42&0.57&0.62&0.51&0.55&0.58&0.41\\
\textbf{Can 10 Yr Future}&0.55&0.70&0.80&0.72&0.81&0.80&0.88\\
\textbf{Euro Bund Future}&0.34&0.42&0.43&0.28&0.33&0.45&0.56\\
\textbf{AUD 10 Yr Future}&0.58&0.54&0.54&0.49&0.63&0.51&0.43\\
\textbf{AU 1-3 year Future}&0.56&0.70&0.59&0.45&0.60&0.45&0.60\\
\textbf{AUD/USD}&0.71&0.70&0.55&0.49&0.51&0.42&0.48\\
\textbf{CAD/USD}&0.46&0.55&0.50&0.46&0.56&0.61&0.71\\
\textbf{EUR/USD}&0.32&0.53&0.36&0.26&0.44&0.47&0.57\\
\textbf{JPY/USD}&0.56&0.57&0.49&0.51&0.49&0.50&0.47\\
\end{tabular}
\end{scriptsize}

    \end{adjustwidth}
    \caption[GOF tests of AR(1)-GARCH(1,1) with Student residuals]{\emph{p}-values from the goodness-of-fit tests of AR(1)-GARCH(1,1) with Student residuals on the marginal processes. The number of bootstrapped samples is $N=100$.}
    \label{ar1garch11_t}
\end{table}
\begin{table}[htb]
    \begin{adjustwidth}{-1in}{-1in}
    \centering
    \begin{scriptsize}
\begin{tabular}{lccccccc}
&\textbf{Feb06}&\textbf{Mar07}&\textbf{Apr08}&\textbf{Apr09}&\textbf{May10}&\textbf{Jun11}&\textbf{Jun12}\\\hline
\textbf{MV Gaussian}&0.00&0.00&0.00&0.00&0.00&0.00&0.00\\
\textbf{AR(1)-GARCH(1,1) \& Gaussian copula}&0.13&0.04&0.00&0.00&0.00&0.00&0.00\\
\textbf{AR(1)-GARCH(1,1) \& Student copula}&0.64&0.17&0.34&0.21&0.17&0.23&0.19\\
\textbf{AR(1)-GARCH(1,1) \& Clayton copula}&0.00&0.00&0.00&0.00&0.00&0.00&0.00\\
\textbf{AR(1)-GARCH(1,1) \& Frank copula}&0.00&0.00&0.00&0.00&0.00&0.00&0.00\\
\textbf{AR(1)-GARCH(1,1) \& Gumbel copula}&0.00&0.00&0.00&0.00&0.00&0.00&0.00\\
\end{tabular}
\end{scriptsize}

    \end{adjustwidth}
    \caption[GOF test of dependence copulas]{\emph{p}-values from the goodness-of-fit tests of copulas. The marginal processes were modeled with AR(1)-GARCH(1,1) with Student residuals. The number of bootstrapped samples is $N=100$.}
    \label{copulas}
\end{table}

\subsection{Results}

Tables~\ref{resultsETE_complete},~\ref{resultsVaR_complete}~and~\ref{resultsTCE_complete}
exhibit the results of the backtests
over the entire sample for the three risk measures of interest.
The copula-based multivariate dynamic approach always had the
lowest frequency of margin
calls, while maintaining similar risk measures realizations.
\begin{table}[p]
    \begin{adjustwidth}{-1in}{-1in}
    \centering
    \begin{small}\begin{tabular}{ l c c c }
&\textbf{Naive}&\textbf{MV Gaussian}&\textbf{AR-GARCH \& t copula}\\
\textbf{Avg. daily tracking error (\% collateral)}&0.52&0.52&0.52\\
\textbf{\# of margin calls (out of 1614 days)}&110&99&85\\
\textbf{Frequency of margin call}&0.07&0.06&0.05\\
\end{tabular}
\end{small}

    \end{adjustwidth}
    \caption[Results of the backtests - Minimized ETE - 2003-2012]{Results of the backtests for the three strategies for the complete dataset period (November 2003 to March 2012) when the optimization objective is set to minimize the tracking error.}
    \label{resultsETE_complete}
\end{table}
\begin{table}[p]
    \begin{adjustwidth}{-1in}{-1in}
    \centering
    \begin{small}\begin{tabular}{ l c c c }
&\textbf{Naive}&\textbf{MV Gaussian}&\textbf{AR-GARCH \& t copula}\\
\textbf{Realized daily VaR (\% collateral)}&1.12&1.20&1.13\\
\textbf{\# of margin calls (out of 1614 days)}&110&90&89\\
\textbf{Frequency of margin call}&0.07&0.06&0.06\\
\end{tabular}
\end{small}

    \end{adjustwidth}
    \caption[Results of the backtests - Minimized VaR - 2003-2012]{Results of the backtests for the three strategies for the complete dataset period (November 2003 to March 2012) when the optimization objective is set to minimize the Value-at-Risk.}
    \label{resultsVaR_complete}
\end{table}
\begin{table}[p]
    \begin{adjustwidth}{-1in}{-1in}
    \centering
    \begin{small}\begin{tabular}{ l c c c }
&\textbf{Naive}&\textbf{MV Gaussian}&\textbf{AR-GARCH \& t copula}\\
\textbf{Avg. daily tail loss (\% collateral)}&-1.75&-1.79&-1.79\\
\textbf{\# of margin calls (out of 1614 days)}&110&99&83\\
\textbf{Frequency of margin call}&0.07&0.06&0.05\\
\end{tabular}
\end{small}

    \end{adjustwidth}
    \caption[Results of the backtests - Minimized TCE - 2003-2012]{Results of the backtests for the three strategies for the complete dataset period (November 2003 to March 2012) when the optimization objective is set to maximize the Tail Conditional Expectation}
    \label{resultsTCE_complete}
\end{table}
The quality of a risk management strategy is known during rough,
volatile markets environments. We thus observed how our proposed
model fared during the 2008-2009 financial crisis. We chose the period
form March $17^{th}$ 2008, the fall of Bear Stearns, followed not long
after by Lehman Brothers and American Insurance Group (AIG), to February $17^{th}$ 2009, when
the American Recovery and Reinvestment Act was passed. As
we now know this date did not mark the end of the crisis, but it does
represent a milestone when volatility in the futures and exchange rates
markets declined. Tables~\ref{resultsETE_crisis}, \ref{resultsVaR_crisis} and
\ref{resultsTCE_crisis} exhibit the results of the backtests
over this critical period. During the crisis, the copula-based multivariate
dynamic model clearly outperformed the two other strategies. Though it did
breach the limits on the frequency of margin calls, it did so in a way much less
drastic than the naive and Gaussian strategies, while realizing similar risk statistics.
\begin{table}[p]
    \begin{adjustwidth}{-1in}{-1in}
    \centering
    \begin{small}\begin{tabular}{ l c c c }
&\textbf{Naive}&\textbf{MV Gaussian}&\textbf{AR-GARCH \& t copula}\\
\textbf{Avg. daily tracking error (\% collateral)}&0.75&0.78&0.79\\
\textbf{\# of margin calls (out of 220 days)}&38&47&20\\
\textbf{Frequency of margin call}&0.17&0.21&0.09\\
\end{tabular}
\end{small}

    \end{adjustwidth}
    \caption[Results of the backtests - Minimized ETE - 2008-2009]{Results of the backtests for the three strategies for the crisis subperiod (March 2008 to February 2009) when the optimization objective is set to minimize the tracking error.}
    \label{resultsETE_crisis}
\end{table}
\begin{table}[p]
    \begin{adjustwidth}{-1in}{-1in}
    \centering
    \begin{small}\begin{tabular}{ l c c c }
&\textbf{Naive}&\textbf{MV Gaussian}&\textbf{AR-GARCH \& t copula}\\
\textbf{Realized daily VaR (\% collateral)}&2.09&2.01&2.13\\
\textbf{\# of margin calls (out of 220 days)}&38&38&22\\
\textbf{Frequency of margin call}&0.17&0.17&0.10\\
\end{tabular}
\end{small}

    \end{adjustwidth}
    \caption[Results of the backtests - Minimized VaR - 2008-2009]{Results of the backtests for the three strategies for the crisis subperiod (March 2008 to February 2009) when the optimization objective is set to minimize the Value-at-Risk.}
    \label{resultsVaR_crisis}
\end{table}
\begin{table}[p]
    \begin{adjustwidth}{-1in}{-1in}
    \centering
    \begin{small}\begin{tabular}{ l c c c }
&\textbf{Naive}&\textbf{MV Gaussian}&\textbf{AR-GARCH \& t copula}\\
\textbf{Avg. daily tail loss (\% collateral)}&-2.81&-2.70&-2.79\\
\textbf{\# of margin calls (out of 220 days)}&38&43&15\\
\textbf{Frequency of margin call}&0.17&0.20&0.07\\
\end{tabular}
\end{small}

    \end{adjustwidth}
    \caption[Results of the backtests - Minimized TCE - 2008-2009]{Results of the backtests for the three strategies for the crisis subperiod (March 2008 to February 2009) when the optimization objective is set to maximize the Tail Conditional Expectation.}
    \label{resultsTCE_crisis}
\end{table}

