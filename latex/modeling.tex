\section{Copula-based multivariate dynamic models}
\label{modelling}
In order to develop a cash management strategy which strikes an optimal balance
between the opposing goals of minimizing foreign exchange risk and minimizing
the cost associated with margin calls, it is first necessary to choose a model of underlying asset prices
and exchange rates movements. It is now widely accepted that proper modeling
is robust to volatility clustering and excess kurtosis. That is, the volatility in
financial time series is not constant across time but periods of relatively high volatility
tend to cluster together. Furthermore, the probability of extreme events in the joint
distributions of multivariate financial time series innovations is higher than predicted
under a Gaussian model. The latter phenomenon is also known as ``fat tails''.
Refer to \cite{pagan96, cont01}
for evidence of stylized facts of financial time series and more precisely to
\cite{casper94, guillaume97} for evidence of such facts in exchange rate returns.

We propose the use of a dynamic multivariate discrete stochastic volatility model
with a copula-based dependence structure. This model has been successfully applied
to exchange rate returns in the literature \citep{chenfan06,Remillard11,patton06}. The
multivariate time series $\mathbf{X}_t$, $t\geq 1$ has $D$ dimensions and is given by
\begin{equation}
\label{dynmodel}
    X_{i,t}=\mu_t(\boldsymbol\theta_i)+h_t(\boldsymbol\theta_i)^{1/2} \
    \epsilon_{i,t},
\end{equation}
where $i=1,\ldots,D$
and innovations $\epsilon_{1,t},\ldots,\epsilon_{D,t}$ are \emph{i.i.d.} with
copula distribution function $C$.

We know from Sklar theorem \citep{sklar59} that, given that $K$ is continuous, there
exists a unique copula $C$ such that
\begin{equation}
\label{sklar}
     K(x_1,\ldots,x_D)=C_{\boldsymbol\theta}(F_1(x_1),\ldots,F_D(x_D)),
\end{equation}
where the $F_i$ are the cumulative distribution
functions of the marginal distributions $X_i$ and $C_{\boldsymbol\theta}$ is the copula
function with parameter(s) $\boldsymbol\theta$.

An interesting property of copulas is that the dependence between the variables is
encapsulated in the copula function and is independent of the marginal distribution
functions chosen.

We model the marginal distributions of the financial time series
with AR(k)-GARCH(p,q) models \citep{bollerslev86}. We chose this
model because we believe it offers the best equilibrium between
sophistication and parsimony. Their formulation is given by
\begin{equation}
\label{drift}
    \mu_t(\boldsymbol\theta_i)=\kappa_{\mu,i}+ \sum_{j=1}^k \gamma_{i,j} \
    x_{i,t-j},
\end{equation}
and
\begin{equation}
\label{volatility}
    h_t(\boldsymbol\theta_i)=\kappa_{h,i} + \sum_{j=1}^q \alpha_{i,j} \ \epsilon_{i,t-j}^2 +  \sum_{j=1}^p \beta_{i,j} \
    h_{t-j}(\boldsymbol\theta_i).
\end{equation}

We tested the goodness-of-fit of common elliptical (Gaussian and
Student) and Archimedean (Clayton, Frank, Gumbel)  copulas on the
standardized residuals of the AR-GARCH margins processes emerging
from the dataset described in section \ref{dataset}. Their
distributions are given below, and the recipe of the parametric
bootstrapping goodness-of-fit tests is given in Appendix \ref{GOF}.

The Gaussian copula with dependence parameter matrix $\Sigma$ is given by
\begin{equation}
\label{gaussiancopuladist}
C_{\Sigma}(\mathbf{u})=\Phi_{\Sigma}\left(\Phi^{-1}(u_1), \ \ldots
\, \Phi^{-1}(u_D)\right),
\end{equation}
where $\Phi^{-1}(\cdot)$ is the inverse cumulative distribution function of a standard normal, $\Phi_{\Sigma}(\cdot)$ is the joint
cumulative distribution function of a multivariate normal distribution with zero means and covariance matrix $\Sigma$. Its density is given by
\begin{displaymath}
c_{\Sigma}(\mathbf{u})=\left| \Sigma \right|^{-\frac{1}{2}} \
\exp\left(-\frac{1}{2}  \left(\begin{array}{c} \Phi^{-1}(u_1)\\
\vdots \\  \Phi^{-1}(u_D) \end{array}  \right)^T
(\Sigma^{-1}-\mathbf{I}) \left(\begin{array}{c} \Phi^{-1}(u_1)\\
\vdots \\  \Phi^{-1}(u_D) \end{array}  \right)\right).
\end{displaymath}

The \emph{D}-dimensional Student copula distribution as given by \citep{mcneil05}:
\begin{equation}
\label{tcopuladist} C_{\Sigma,\nu}(\mathbf{u}) =
\int_{-\infty}^{t_{\nu}^{-1}(u_1)}  \ldots
\int_{-\infty}^{t_{\nu}^{-1}(u_D)} \frac{\Gamma\left(
\frac{\nu+D}{2}\right)}{\Gamma \left( \frac{\nu}{2} \right)
\sqrt{(\pi \nu)^D |\Sigma|}} \left( 1+\frac{\mathbf{x}' \Sigma^{-1}
\mathbf{x}}{\nu}\right)^{-\frac{\nu+D}{2}} d\mathbf{x},
\end{equation}
where
$t_{\nu}^{-1}()$ is the quantile function of a standard univariate Student distribution with $\nu$
degrees of freedom. Its density can be derived to be
\begin{equation}
\label{tcopuladens}
c_{\Sigma,\nu}(\mathbf{u})=\frac{f_{\Sigma,\nu}(t_{\nu}^{-1}(u_1),\ldots,t_{\nu}^{-1}(u_D))}{\prod_{i=1}^D
f_{\nu}(t_{\nu}^{-1}(u_i))},
\end{equation}
where $f_{\Sigma,\nu}$ is the joint density of a D-dimensional random vector from a multivariate
Student distribution with $\nu$ degrees of freedom and covariance matrix $\Sigma$ and
$f_{\nu}$ is the density of a univariate Student distribution with $\nu$ degrees of freedom.
 This copula not only captures excess kurtosis but was also shown to accurately model
the dependence between financial time series in recent literature \citep{chenfan06,fisher09}.

Archimedean copulas are characterized by a single dependence parameter and the following representation:
\begin{displaymath}
C(u_1, \ u_2, \ \ldots \ ,
u_D)=\psi(\psi^{-1}(u_1)+\ldots+\psi^{-1}(u_D)),
\end{displaymath}
where $\psi(\cdot)$ is the \emph{generator} of the copula. The
generators for the three Archimedean copulas tested in the applied
section of this thesis are displayed in table \ref{archi}.
\begin{table}[htb]
    \centering
    \begin{footnotesize}
    \begin{tabular}{c c c c}
    \textbf{Family}&\textbf{Generator} $\boldsymbol\psi\mathbf{(t)}$&\textbf{Parameter}&$\mathbf{G}$\textbf{-distribution}\\
    Clayton&$(1+t)^{-1/\theta}$&$0 < \theta < \infty$&Gamma$(1/\theta,1)$\\
    Frank&$-\frac{1}{\theta}\log(1-(1-e^{-\theta}) e^{-t})$&$0 < \theta < \infty$&Log series with $\alpha=(1-e^{-\theta})$\\
    Gumbel&$\exp(-t^{1/\theta})$&$1 \leq \theta < \infty$&Stable$(1/\theta,1,(\cos(\pi/(2\theta)))^{\theta},0)$\\
    \end{tabular}
    \end{footnotesize}
    \caption[Generators of selected Archimedean copulas]{Generators of selected Archimedean copulas. G-distribution refers to the
distribution that has as Laplace transform the generator $\psi(\cdot)$.}
    \label{archi}
\end{table}


The parameters of the marginal processes and dependence copula can
be estimated with the maximum likelihood estimation (MLE) method. An
alternative two-stage estimation method for the Student copula
is proposed by \cite{mcneil05b}.

\cite{chenfan06} showed the important result that applying the copula dependence structure
from equation (\ref{sklar}) to the innovations $\epsilon_{i,t}$ from equation (\ref{dynmodel}) rather
than to the realizations $X_{i,t}$ yields the same results
in terms of estimation of the parameters of the copula. \cite{Remillard11} showed that
the empirical copula and most dependence measures are unaffected as well.

