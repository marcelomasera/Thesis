\section{Conclusion}
\label{conclusion}
We met three objectives in this work; first, we proposed a model that
better fits high-dimensional multivariate financial time-series than the classical
Gaussian model with the copula-based multivariate dynamic model. Second, we used
recent advances in absolute goodness-of-fit tests to show the appropriateness of the
chosen model from a statistical point of view. Third, we used the proposed model
to solve a problem encountered by managers
whose portfolio contains multiple assets on different exchanges and/or in different
currencies, that is, how to balance the opposing nuisances
of margin calls and foreign exchange risk on collateral posted in a currency
different from the participant's benchmark currency. The solution lies in calibrating
the model proposed in this work (or another robust model encapsulating stochastic
volatility, excess kurtosis and higher co-moments) to the financial time series of interest.
Then the desired constrained optimization is solved using Monte Carlo methods.

Two potential avenues of future study are improvements in model sophistication
and the development of rigorous models for problems with high numbers of random
variables. On the first front, models
including time-varying copula parameters, regime-switching and/or correlation asymmetry
may yield a better
fit to financial time series. On the second front, portfolios
often have a large number of assets in many different currencies, however the
use of copula becomes exponentially harder when the number of dimension
is above 10. Methods for dimensionality reduction that conserve higher moments
in the factors or efficiency improvements in algorithms for copula use would help
toward accomplishing this objective.
